%!TeX program = xelatex

%%%%%%%%%%%%%%%%%%%%%%%%%%%%%%%%%%%%%%%%%
% Medium Length Professional CV
% LaTeX Template
% Version 2.0 (8/5/13)
%
% This template has been downloaded from:
% http://www.LaTeXTemplates.com
%
% Original author:
% Trey Hunner (http://www.treyhunner.com/)
%
% Important note:
% This template requires the resume.cls file to be in the same directory as the
% .tex file. The resume.cls file provides the resume style used for structuring the
% document.
%
%%%%%%%%%%%%%%%%%%%%%%%%%%%%%%%%%%%%%%%%%

%----------------------------------------------------------------------------------------
%   PACKAGES AND OTHER DOCUMENT CONFIGURATIONS
%----------------------------------------------------------------------------------------

\documentclass{resume} % Use the custom resume.cls style
\usepackage[dvipsnames]{xcolor}
\definecolor{accent_light}{Hsb}{220, 0.95, 0.9}
\definecolor{accent_dark}{Hsb}{220, 0.95, 0.3}

\usepackage[misc]{ifsym} % loads letter symbol
\usepackage{fontawesome} % loads other symbols
\usepackage{graphicx}    % for the \rotatebox command
\usepackage{datetime}    % to adjust date format

\usepackage[colorlinks=true, urlcolor=accent_light]{hyperref}
\usepackage[left=0.75in,top=0.6in,right=0.75in,bottom=0.75in]{geometry} % Document margins
\newcommand{\tab}[1]{\hspace{.2667\textwidth}\rlap{#1}}
\newcommand{\itab}[1]{\hspace{0em}\rlap{#1}}
\usepackage{multicol} % for multicolumn environment (references)

\usepackage{changepage} % To add indent to abstracts
\renewenvironment{abstract}{\begin{adjustwidth}{1em}{}}{\end{adjustwidth}}

% \usepackage{bold-extra}
\usepackage[T1]{fontenc}
\usepackage{fontspec}

\setmainfont{Montserrat ExtraLight}[BoldFont = Montserrat Regular]
\newfontfamily\scdfamily{Cinzel Decorative}[BoldFont = CinzelDecorative-Bold]

\newfontfamily\scfamily{Cinzel}[
    FontFace = {sb}{n}{Font = * SemiBold}
    ]

\newcommand\cinzel[1]{{\scfamily #1}}
\newcommand\cinzeldeco[1]{{\scdfamily #1}}

\DeclareRobustCommand{\sbseries}{\fontseries{sb}\selectfont}
\DeclareTextFontCommand{\textsb}{\sbseries}


\newdateformat{monthdayyeardate}{%
  \monthname[\THEMONTH]~\THEDAY, \THEYEAR}

% \renewcommand{\faPhone}{\rotatebox[origin=c]{50}{\faPhone}}

% \name{\color{accent_dark}\fontsize{32}{1}\bfseries\cinzel{\fontsize{36}{1}{\cinzeldeco{R}}iccardo \cinzeldeco{A}ntonio \cinzeldeco{C}ioffi}} % Your name
\name{\color{accent_dark}\fontsize{24}{1}\bfseries\cinzel{Riccardo Antonio Cioffi}} % Your name
\address{\href{mailto:rcioffi@princeton.edu}{{\footnotesize{\Letter}} rcioffi@princeton.edu} \\ %
         \href{tel:+16094235568}{{\footnotesize{\faPhone}} +1 (609) 423-5568} \\ %
         \href{https://www.rcioffi.com}{{\footnotesize{\faExternalLink}} www.rcioffi.com}} % Your phone number and email
\address{Department of Economics \\ Julis Romo Rabinowitz Building \\ Princeton University} % 



\renewenvironment{rSection}[1]{
\sectionskip
{\color{accent_dark}\textsb{\cinzel{\Large{\textsc{#1}}}}}
\normalsize
\sectionlineskip
{\color{accent_dark}\vspace{-2.5pt}\hrule height 1.5pt}
\begin{list}{}{
\setlength{\leftmargin}{1.5em}
}
\item[]
}{
\end{list}
}


\begin{document}
{\footnotesize
Placement Director: Gianluca Violante \hfill \href{mailto:glv2@princeton.edu}{{\footnotesize{\Letter}} violante@princeton.edu}   \hspace{2mm} \href{tel:+16094235568}{{\footnotesize{\faPhone}} +1 (609) 258-4003} \\
Graduate Administrator: Laura Hedden \hfill  \href{mailto:lhedden@princeton.edu}{{\footnotesize{\Letter}} lhedden@princeton.edu}  \hspace{2mm} \href{tel:+16094235568}{{\footnotesize{\faPhone}} +1 (609) 258-4006}\\
}
\vspace*{-2em}


%----------------------------------------------------------------------------------------
%   EDUCATION SECTION
%----------------------------------------------------------------------------------------

\begin{rSection}{Education}


{{\sc \textbf{Ph.D. in Economics}} \hfill {\em 2015 -- Present (Expected May 2022)} 
\\ Princeton University \hfill} \\[0.25ex]
%
{{\sc \textbf{M.A. in Economics}} \hfill {\em 2015 -- 2017} 
\\ Princeton University \hfill} \\[0.25ex]
%
{{\sc \textbf{M.A. in Economics and Finance}} \hfill {\em 2012 -- 2014} 
\\ University of Naples Federico II \hfill} \\[0.25ex]
%
{{\sc \textbf{B.A. in Economics}} \hfill {\em 2009 -- 2012} 
\\ University of Naples Federico II \hfill}

%Minor in Linguistics \smallskip \\
%Member of Eta Kappa Nu \\
%Member of Upsilon Pi Epsilon \\

\end{rSection}\vspace*{-1.5ex}

%----------------------------------------------------------------------------------------
%   Research Interests
%----------------------------------------------------------------------------------------

\begin{rSection}{Research Interests}

Macroeconomics, Finance, Household Finance, Inequality

\end{rSection}\vspace*{-1.5ex}

%----------------------------------------------------------------------------------------
%   REFERENCES
%----------------------------------------------------------------------------------------

\begin{rSection}{References}
\vspace*{-6ex}
\begin{multicols}{3}
\section*{}
{\sc\textbf{Gianluca Violante}} \newline
Department of Economics\newline
Princeton University\newline
\href{tel:+16094235568}{{\footnotesize{\faPhone}} +1 (609) 258-4003}\newline
\href{mailto:violante@princeton.edu}{{\footnotesize{\Letter}} violante@princeton.edu}\\
\section*{}
{\sc\textbf{Richard Rogerson}}\newline
Department of Economics\newline
Princeton University\newline
\href{tel:+16092584839}{{\footnotesize{\faPhone}} +1 (609) 258-4839}\newline
\href{mailto:rdr@princeton.edu}{rdr@princeton.edu \Letter}\newline
\section*{}
{\sc\textbf{Benjamin Moll}}\newline
Department of Economics\newline
London School of Economics\newline
\href{tel:+442079557507}{{\footnotesize{\faPhone}} +44 20-7955-7507}\newline
\href{mailto:b.moll@lse.ac.uk}{b.moll@lse.ac.uk \Letter}\newline
\end{multicols}
\end{rSection}\vspace{-6ex}


%----------------------------------------------------------------------------------------
%   Working Papers
%----------------------------------------------------------------------------------------

\begin{rSection}{Job Market Paper}

{\sc \textbf{Heterogeneous Risk Exposure and the Dynamics of Wealth Inequality}} \\[-1.5em]
{\begin{abstract}
    \footnotesize 
    In this paper I show that, in the presence of heterogeneous risk exposure along the wealth distribution, aggregate movements in asset returns generate fluctuations in wealth inequality.
    To match both the level and the dynamics of wealth inequality, it therefore suffices to have a model consistent with the observed portfolio-choice behavior along the wealth distribution coupled with realistic features for asset returns.
    I then propose a model where - just like in the data - as households get wealthier they shift their portfolios away from safe assets, first towards housing, and then towards equity.
    The dual role of housing as an investment asset and a necessary good is crucial to generate portfolio shares in line with the data.
    Finally, I show that fluctuations in asset returns generate large swings in inequality over time that replicate the variation observed in U.S. data.
\end{abstract}}
\vspace*{0.5em}

\end{rSection}\vspace*{-1.5ex}

% %----------------------------------------------------------------------------------------
% %   Work in progress
% %----------------------------------------------------------------------------------------

\begin{rSection}{Other Papers}

% {Wealth Inequality and Asset Returns: the Role of Heterogeneous Exposure to Aggregate Risk(s)}\hfill {}
% {\sc \textbf{The Direct Effect of Wealth on Portfolio Choice: Evidence from Norway}}\hfill {}

{\sc \textbf{When Does Wealth Inequality Matter for Asset Pricing?}} \\[-1.5em]
{\begin{abstract}
    \footnotesize
    In this paper I show that, contrary to conventional wisdom, the wealth distribution does matter for the determination of asset prices.
    I do so by showing that, in a model in which households' equity share is increasing in wealth, approximate aggregation does not hold and households make systematic errors when trying to forecast prices ignoring wealth inequality.
    In order to understand the effect of inequality on asset prices, I solve a two-assets general equilibrium model of wealth inequality and use recent advances from scientific machine learning to extend the algorithm in Villaverde et al. (2020) to solve systems of neural stochastic differential equations for the aggregate states.
    Finally, I look at how the introduction of such GE feedback between wealth inequality and asset prices changes our understanding of the effects of government policy.
\end{abstract}}
\vspace*{0.5em}

\newpage

{{\sc \textbf{Wealth Inequality at the Top: Down to the Roots}} \hfill {\footnotesize \em (joint with G. Sorg-Langhans and M. Vogler)}} \\[-1.5em]
{\begin{abstract}
    \footnotesize
    Multiple theories of inequality compete to explain U.S. wealth inequality and the share of wealth held by the top one percent. To what extent does it matter which of these models we rely on? In this paper we analyze the responses of the different theories to a host of policy experiments. 
    To this end, we form a quantitative model that nests the competing channels and assesses the effects of policy experiments by sequentially shutting off all but one of these model mechanisms. Our model is directly calibrated on the wealth distribution which allows us to starkly contrast the different theories and clearly understand the mechanisms at work. We find stark differences in predictions across channels for a given policy experiment, indicating that, by choosing a particular mechanism, researchers might already predetermine the outcome of their policy experiments.
\end{abstract}}
\vspace*{0.5em}

\end{rSection}\vspace*{-2ex}

\begin{rSection}{Works in Progress} 
    {The Direct Effect of Wealth on Portfolio Choice: Evidence from Norway}\hfill {}    
\end{rSection}


%----------------------------------------------------------------------------------------
%   VISITING SECTION
%----------------------------------------------------------------------------------------

\begin{rSection}{Visiting Positions}

{{\sc \textbf{Federal Reserve Board of Governors}} \hfill {\sc \textbf{Washington, DC}}
\\ Dissertation Fellow \hfill {\em Summer 2021}}\\[0.25ex]
%
{{\sc \textbf{Federal Reserve Bank of St. Louis}} \hfill {\sc \textbf{St. Louis, MO}}
\\ Dissertation Intern (workshop due to COVID-19) \hfill {\em Summer 2020}}\\[0.25ex]
%
{{\sc \textbf{Statistics Norway}} \hfill {\sc \textbf{Oslo, Norway}}
\\ Visiting Scholar \hfill {\em 2018 -- Present}}\\[0.25ex]
%
{\sc \textbf{Capital Markets Cooperative Research Centre}} \hfill {\sc \textbf{Sydney, Australia}}
\\ Visiting Scholar \hfill {\em Spring 2015}

% {\bf Goethe University} \hfill {\bf Frankfurt, Germany}
% \\ Erasmus Exchange Student \hfill {\em Fall 2013}

\end{rSection}\vspace*{-1.5ex}

%----------------------------------------------------------------------------------------
%   TEACHING EXPERIENCE SECTION
%----------------------------------------------------------------------------------------

\begin{rSection}{Teaching Experience}

{{\sc \textbf{Graduate -- High Performance Computing in Economics}} \\
{Instructor} \hfill {\em 2019 -- 2021}} \\[0.25ex]
%
{{\sc \textbf{Undergraduate -- Intermediate Macroeconomics}} \\
{Teaching Assistant -- Gianluca Violante} \hfill {\em Spring 2018}}\\[0.25ex]
%
{{\sc \textbf{Undergraduate -- Introductory Microeconomics}} \\
{Teaching Assistant -- Harvey Rosen} \hfill {\em Fall 2018} \\
{Teaching Assistant -- Henry Farber} \hfill {\em Fall 2017}}

\end{rSection}\vspace*{-1.5ex}

%----------------------------------------------------------------------------------------
%   RESEARCH EXPERIENCE SECTION
%----------------------------------------------------------------------------------------

\begin{rSection}{Research Experience} \itemsep - 2pt

{Research Assistant -- Gianluca Violante} \hfill {\em Princeton University, 2018 -- 2019} \\
{Research Assistant -- Benjamin Moll} \hfill {\em Princeton University, 2016 -- 2018} \\
{Research Assistant -- Oleg Itskhoki} \hfill {\em Princeton University, 2016} \\
{Research Assistant -- Marco Pagano} \hfill {\em University of Naples, 2014 -- 2015}

\end{rSection}\vspace*{-1.5ex}

%----------------------------------------------------------------------------------------
%   HONORS AND AWARDS
%----------------------------------------------------------------------------------------

\begin{rSection}{Honors and Awards} \itemsep -2pt

{Graduate Fellowship -- Princeton University}\hfill {\em 2015 - 2021} \\
{Griswold Center for Economy Policy Studies Fellowship -- Princeton University}\hfill {\em 2019 - 2020} \\
{Marco Fanno Scholarship -- UniCredit \& Universities Foundation}\hfill {\em 2014 - 2015} \\
{``Messaggeri della Conoscenza'' Program Scholarship}\hfill {\em 2015} \\
{Best Master Student -- University of Naples} \hfill {\em 2014}
\end{rSection}\vspace*{-1.5ex}


%----------------------------------------------------------------------------------------
%   PROGRAMMING SKILLS SECTION
%----------------------------------------------------------------------------------------

\begin{rSection}{Programming Skills}

Julia, R, Matlab, Stata, \LaTeX

\end{rSection}\vspace*{-1.5ex}


%----------------------------------------------------------------------------------------
%   LANGUAGES SECTION
%----------------------------------------------------------------------------------------

\begin{rSection}{Languages}

Italian (native), English

\end{rSection}\vspace*{-1.5ex}

%----------------------------------------------------------------------------------------
\vspace{1cm}
\begin{flushright}
    \footnotesize
    Last updated: \monthdayyeardate\today
\end{flushright}



\end{document}
